% This is samplepaper.tex, a sample chapter demonstrating the
% LLNCS macro package for Springer Computer Science proceedings;
% Version 2.21 of 2022/01/12
%
\documentclass[runningheads]{llncs}
%
\usepackage[T1]{fontenc}
% T1 fonts will be used to generate the final print and online PDFs,
% so please use T1 fonts in your manuscript whenever possible.
% Other font encondings may result in incorrect characters.
%
\usepackage{graphicx}
% Used for displaying a sample figure. If possible, figure files should
% be included in EPS format.
%
% If you use the hyperref package, please uncomment the following two lines
% to display URLs in blue roman font according to Springer's eBook style:
%\usepackage{color}
%\renewcommand\UrlFont{\color{blue}\rmfamily}
%\urlstyle{rm}
%
\usepackage[romanian]{babel}
\usepackage{url}
\usepackage{hyperref}
\begin{document}
%
\title{Problema Rucsacului - Knapsack Problem}
%
%\titlerunning{Abbreviated paper title}
% If the paper title is too long for the running head, you can set
% an abbreviated paper title here
%
\author{Gabriel-Ioan PAVEL \and
Andrei RUSĂNESCU \and
Amir FALLAH-MIRZAEI}
%
\institute{Universitatea de Știință și Tehnologie POLITEHNICA București}
%
\maketitle              % typeset the header of the contribution
%
\begin{abstract}
Acest raport reprezintă analiza câtorva algoritmi ce rezolvă Problema Rucsacului

\end{abstract}
%
%
%
\section{Introducere}
Problema Rucsacului reprezintă problema alocării și gestionării de resurse limitate,
unde ținta este obținerea unui profit maxim.

\subsubsection{Aplicații practice} Printre domeniile în care apare Problema
Rucsacului se numără: găsirea unei metode cât mai puțin risipitoare de a tăia
materii prime, selecția investițiilor și a portofoliilor, selectarea activelor
pentru securitizare garantată cu active etc.

\section{Demonstrație NP-Hard}
%TODO: Amir

\section{Prezentarea algoritmilor}
%TODO: Gebi

\section{Evaluare}
%TODO: Andrei + Gebi

\section{Concluzii}
%TODO: Gebi

%
% ---- Bibliography ----
%
% BibTeX users should specify bibliography style 'splncs04'.
% References will then be sorted and formatted in the correct style.
%
% \bibliographystyle{splncs04}
% \bibliography{mybibliography}
%
\begin{thebibliography}{8}
\bibitem{ref_article1}
Pavel, G.-I.: \emph{Knapsack problem}, \url{https://en.wikipedia.org/wiki/Knapsack_problem} [Accesat la 04/01/2025].

\end{thebibliography}
\end{document}